\documentclass[letterpaper]{article}

\usepackage{graphics}
\usepackage{graphicx}
\usepackage{amsmath}

\title{Notes on an axisymmetric global stability code for the Princeton MRI experiment}
\author{Austin Roach}


\begin{document}
\maketitle{}

\section{Introduction}

Recently, nonaxisymmetric MHD modes have been observed in the
Princeton MRI experiment$^1$.  Previous computational work in support
of the experiment has been almost exclusively axisymmetric.  This work
will follow a similar approach to Goodman and Ji's examination of
unstable axisymmetric global linear modes in a cylindrical system$^2$,
but will extend the analysis to allow examination of nonaxisymmetric
eigenmodes, eigenmodes with non-ideal Couette rotation profiles, and
damped eigenmodes.

\section{Linearized Equations}

We start with the equations of incompressible MHD in CGS units:

\begin{align}
&\dot{{\bf B}} + {\bf v}\cdot\nabla{\bf B} - {\bf B}\cdot\nabla{\bf v}=\eta\nabla^2 {\bf B}
\\
&\dot{{\bf v}}+{\bf v}\cdot\nabla{\bf v}+\frac{1}{\rho}\nabla{P}-\frac{{\bf B}\cdot\nabla{\bf B}}{4\pi\rho}=\nu\nabla^2{\bf v}
\\
&\nabla\cdot{\bf B}=0
\\
&\nabla\cdot{\bf v}=0
\end{align}

Here $P$ is defined as the sum of the hydrodynamic and magnetic
pressures, $P=p+{\bf B}^2/8\pi$.We assume a background state with
zeroth-order ${\bf B}=B_0 \hat{z}$ and ${\bf
  v}=r\Omega(r)\hat{\theta}$.  First order quantities will be denoted
by a $\delta$.

We will define two radial derivative operators, $\partial_r \equiv
\frac{\partial}{\partial r}$ and $\partial_{r}^\dagger \equiv
\frac{\partial}{\partial r}+\frac{1}{r}$.  So the combination
$\partial_r \partial_r^\dagger = \frac{\partial^2}{\partial
  r^2}-\frac{1}{r^2}+\frac{1}{r}\partial_r$.

\subsection{Radial induction equation}

The radial component of the induction equation is
\begin{align}
\dot{B_r} + v_r \partial_r B_r + \frac{v_\theta}{r}\partial_\theta B_r + v_z \partial_z B_r - B_r \partial_r v_r -\frac{B_\theta}{r}\partial_\theta v_r - B_z \partial_z v_r
\\ \nonumber
 = \eta\left(\nabla^2 B_r -\frac{2}{r^2}\partial_\theta B_\theta - \frac{1}{r^2}B_r\right)
\end{align}
After linearization, keeping only 1st-order quantities, this becomes
\begin{equation}
\delta \dot{B_r} + \Omega \partial_\theta \delta B_r - B_0 \partial_z \delta v_r = \eta\left[\left(\partial_r \partial_r^\dagger + \frac{1}{r^2}\partial_\theta^2 + \partial_z^2\right)\delta B_r - \frac{2}{r^2}\partial_\theta \delta B_\theta\right]
\end{equation}

\subsection{Azimuthal induction equation}
The azimuthal component of the induction equation is
\begin{align}
\dot{B_\theta}+v_r \partial_r B_\theta + \frac{v_\theta}{r}\partial_\theta B_\theta + v_z \partial_z B_\theta + \frac{v_\theta}{r}B_r - B_r\partial_r v_\theta - \frac{1}{r}B_\theta \partial_\theta v_\theta
\\ \nonumber
 - B_z \partial_z v_\theta - \frac{v_r}{r}B_\theta = \eta\left(\nabla^2 B_\theta + \frac{2}{r^2}\partial_\theta B_r - \frac{1}{r^2}B_\theta \right)
\end{align}
After linearization, this becomes
\begin{align}
\delta \dot{B_\theta} + \Omega \partial_\theta \delta B_\theta - B_0 \partial_z \delta v_\theta - \delta B_r r \partial_r \Omega=\eta\left[\left(\partial_r \partial_r^\dagger + \frac{1}{r^2}\partial_\theta^2 + \partial_z^2\right)\delta B_\theta + \frac{2}{r^2}\partial_\theta \delta B_r\right]
\end{align}

\subsection{Axial induction equation}
The axial component of the induction equation is
\begin{align}
\dot{B_z} + v_r \partial_r B_z + \frac{v_\theta}{r}\partial_\theta B_z + v_z \partial_z B_z - B_r \partial_r v_z - \frac{1}{r}B_\theta \partial_\theta v_z - B_z \partial_z v_z
\\ \nonumber
=\eta\left[\frac{1}{r}\partial_r + \partial_r^2 + \frac{1}{r^2}\partial_\theta^2 + \partial_z^2\right]B_z
\end{align}
After linearization this becomes
\begin{align}
\delta\dot{B_z} + \Omega\partial_\theta \delta B_z - B_0\partial_z \delta v_z = \eta\left[\partial_r^\dagger \partial_r + \frac{1}{r^2}\partial_\theta^2 + \partial_z^2\right]\delta B_z
\end{align}

\subsection{Radial Euler equation}

The radial component of the Euler equation is
\begin{align}
\dot{v_r} + v_r \partial_r v_r + \frac{v_\theta}{r}\partial_\theta v_r + v_z \partial_z v_r - \frac{v_\theta^2}{r}+\frac{1}{\rho}\partial_r P 
\\ \nonumber 
- \frac{1}{4\pi\rho}\left(B_r \partial_r B_r + \frac{B_\theta}{r}\partial_\theta B_r + B_z \partial_z B_r - \frac{B_\theta^2}{r}\right)
\\ \nonumber
 = \nu\left(\nabla^2 v_r -\frac{2}{r^2}\partial_\theta v_\theta - \frac{1}{r^2}v_r\right)
\end{align}
After linearization this becomes
\begin{align}
\delta\dot{v_r} + \Omega \partial_\theta \delta v_r - 2\Omega\delta v_\theta + \partial_r \frac{\delta P}{\rho} - \frac{B_0}{4\pi\rho}\partial_z \delta B_r
\\ \nonumber
= \nu \left[\left(\partial_r \partial_r^\dagger + \frac{1}{r^2}\partial_\theta^2 + \partial_z^2\right)\delta v_r - \frac{2}{r^2}\partial_\theta \delta v_\theta\right]
\end{align}

\subsection{Azimuthal Euler Equation}
The azimuthal component of the Euler equation is
\begin{align}
\dot{v_\theta} + v_r \partial_r v_\theta + \frac{v_\theta}{r}\partial_\theta v_\theta + v_z \partial_z v_\theta + \frac{v_\theta v_r}{r} + \frac{1}{\rho}\frac{1}{r}\partial_\theta P 
\\ \nonumber
-\frac{1}{4\pi\rho}\left[B_r\partial_r B_\theta + \frac{B_\theta}{r}\partial_\theta B_\theta + B_z \partial_z B_\theta + \frac{B_\theta B_r}{r}\right]
\\ \nonumber
= \nu\left(\nabla^2 v_\theta + \frac{2}{r^2}\partial_\theta v_r - \frac{1}{r^2}v_\theta \right)
\end{align}
After linearization this becomes
\begin{align}
\delta \dot{v_\theta} + \delta v_r \partial_r^\dagger(r\Omega) + \Omega \partial_\theta \delta v_\theta + \frac{1}{r}\partial_\theta \frac{\delta P}{\rho} - \frac{B_0}{4\pi\rho} \partial_z \delta B_\theta
\\ \nonumber
= \nu\left[\left(\partial_r \partial_r^\dagger + \frac{1}{r^2}\partial_\theta^2 + \partial_z^2\right)\delta v_\theta + \frac{2}{r^2}\partial_\theta \delta v_r \right]
\end{align}

\subsection{Axial Euler Equation}
The axial component of the Euler equation is
\begin{align}
\dot{v_z} + v_r \partial_r v_z + \frac{v_\theta}{r}\partial_\theta v_z + v_z \partial_z v_z + \frac{1}{\rho}\partial_z P
\\ \nonumber
-\frac{1}{4\pi\rho}\left[B_r\partial_r B_z + \frac{B_\theta}{r}\partial_\theta B_z + B_z \partial_z B_z \right]
\\ \nonumber
=\nu\left[\frac{1}{r}\partial_r + \partial_r^2 + \frac{1}{r^2}\partial_\theta^2 + \partial_z^2\right]v_z
\end{align}
After linearization this becomes
\begin{align}
\delta\dot{v_z} + \Omega \partial_\theta \delta v_z + \partial_z \frac{\delta P}{\rho} -\frac{B_0}{4\pi\rho} \partial_z \delta B_z = \nu\left[\partial_r^\dagger \partial_r + \frac{1}{r^2}\partial_\theta^2 + \partial_z^2\right]\delta v_z
\end{align}

\subsection{Equations of constraint}

The equation of incompressibility, $\nabla\cdot\vec{v}=0$ is, with no approximation
\begin{align}
\partial_r^\dagger \delta v_r + \frac{1}{r}\partial_\theta \delta v_\theta + \partial_z \delta v_z = 0
\end{align}
and $\nabla\cdot\vec{B}=0$ is
\begin{align}
\partial_r^\dagger \delta B_r + \frac{1}{r}\partial_\theta \delta B_\theta + \partial_z \delta B_z = 0
\end{align}

\section{Forms of perturbations}

Perturbations to the background solution are allowed to have an
arbitrary form with respect to the $\hat{r}$-coordinate.  The
perturbations are assumed to vary periodically in $\hat{z}$ and
$\hat{r}$, with mode numbers $k$ and $m$.  The form of the equations
suggests the relation of the phases of the various field components in
$\hat{z}$.  It does not, however, suggest the phase relation in
$\hat{\theta}$.  Because of that, I suggest the following form for the
perturbations of the field components:
\begin{align*}
&\delta B_r / \sqrt{4\pi\rho} = \operatorname{Re}\{\beta_r (r,t)e^{im\theta}\cos{kz}\},\quad
&\delta v_r = \operatorname{Re}\{\phi_r (r,t) e^{im\theta} \sin{kz}\}
\\
&\delta B_\theta / \sqrt{4\pi\rho} = \operatorname{Re}\{\beta_\theta (r,t)e^{im\theta}\cos{kz}\},\quad
&\delta v_\theta = \operatorname{Re}\{\phi_\theta (r,t) e^{im\theta} \sin{kz}\}
\\
&\delta B_z / \sqrt{4\pi\rho} = \operatorname{Re}\{\beta_z (r,t)e^{im\theta}\sin{kz}\},\quad
&\delta v_z = \operatorname{Re}\{\phi_z (r,t) e^{im\theta} \cos{kz}\}
\\
&\delta P / \rho = \operatorname{Re}\{\Pi (r,t) e^{im\theta} \sin{kz}\}&
\end{align*}

The quantities $\beta$, $\phi$, and $\Pi$ are complex, holding
information about both the magnitude and phase. Following the
convention common in harmonic analysis, the real part of these
variables is evaluated to find physically meaningful
quantities. Because our equations are linear, we are justified in
dropping the `$\mathrm{Re}$' when rewriting the equations in terms of
these perturbations, shown below. Note that we have made use of the
Alfv\'{e}n speed, $v_A = B_0/\sqrt{4\pi\rho}$.

\begin{equation}\label{rind_p}
\dot{\beta_r} + im\Omega\beta_r - kv_A\phi_r = \eta\left[\left(\partial_r \partial_r^\dagger - \frac{m^2}{r^2} - k^2\right)\beta_r - \frac{2im}{r^2}\beta_\theta\right]
\end{equation}
\begin{equation}\label{tind_p}
\dot{\beta_\theta} + im\Omega\beta_\theta - kv_A \phi_\theta - r(\partial_r \Omega)\beta_r = \eta\left[\left(\partial_r \partial_r^\dagger - \frac{m^2}{r^2} - k^2\right)\beta_\theta + \frac{2im}{r^2}\beta_r\right]
\end{equation}
\begin{equation}\label{zind_p}
\dot{\beta_z} + im\Omega\beta_z + kv_A \phi_z = \eta\left[\left(\partial_r^\dagger \partial_r -\frac{m^2}{r^2} - k^2\right)\beta_z\right]
\end{equation}
\begin{equation}\label{reul_p}
\dot{\phi_r} + im\Omega\phi_r - 2\Omega\phi_\theta + \partial_r \Pi + kv_A\beta_r = \nu\left[\left(\partial_r \partial_r^\dagger - \frac{m^2}{r^2} - k^2\right)\phi_r - \frac{2im}{r^2}\phi_\theta\right]
\end{equation}
\begin{equation}\label{teul_p}
\dot{\phi_\theta} + im\Omega\phi_\theta + \phi_r \partial_r^\dagger(r\Omega) + \frac{im}{r}\Pi + kv_A \beta_\theta = \nu\left[\left(\partial_r \partial_r^\dagger - \frac{m^2}{r^2} - k^2\right)\phi_\theta + \frac{2im}{r^2}\phi_r\right]
\end{equation}
\begin{equation}\label{zeul_p}
\dot{\phi_z} + im\Omega{\phi_z} + k\Pi - kv_A \beta_z = \nu\left[\left(\partial_r^\dagger \partial_r - \frac{m^2}{r^2} - k^2\right)\phi_z\right]
\end{equation}
\begin{equation}\label{inc_p}
\partial_r^\dagger \phi_r + \frac{im}{r}\phi_\theta - k\phi_z = 0
\end{equation}
\begin{equation}\label{delB_p}
\partial_r^\dagger \beta_r + \frac{im}{r}\beta_\theta + k\beta_z = 0
\end{equation}

\section{Reduced set of equations}

We can reduce the set of equations by applying the operator
$\partial_r^\dagger$ to Equation~\ref{reul_p}, $(im/r)$
to Equation~\ref{teul_p}, and $-k$ to Equation~\ref{zeul_p},
and then summing the resulting equations.  We can use the equations of
constraint to eliminate many of the terms in the resulting equation.

The following relations will be needed in order to massage some of the
terms into the correct form for elimination by the constraint
equations:
\begin{equation}
\frac{1}{r}\partial_r \partial_r^\dagger = \partial_r^\dagger \partial_r \frac{1}{r} - \frac{2}{r^3} + \frac{2}{r^2}\partial_r
\end{equation}
\begin{equation}
\partial_r^\dagger \frac{1}{r^2} = \frac{1}{r^2} \partial_r^\dagger - \frac{2}{r^3}
\end{equation}

This produces the following constraint equation:
\begin{equation}\label{piconeq}
im\left(\frac{2\Omega}{r}+2\partial_r \Omega\right)\phi_r - \partial_r^\dagger\left(2\Omega\phi_\theta\right)+\left(\partial_r^\dagger \partial_r - \frac{m^2}{r^2}-k^2\right)\Pi = 0
\end{equation}

We will solve the system of equations formed by Equations
\ref{rind_p}, \ref{tind_p}, \ref{reul_p}, \ref{zeul_p}, and
\ref{piconeq} for the quantities $\beta_r$, $\beta_\theta$, $\phi_r$,
$\phi_\theta$, and $\Pi$.  $\beta_z$ and $\phi_z$ can be found from
$\nabla\cdot{\bf B}=0$ and $\nabla\cdot{\bf v}=0$.
\section{Numerics}

Following Goodman and Ji, we will make use of a logarithmic grid in
$r$, equally spaced in $x=\ln{r}$.  The radial derivatives will now be
replaced by $\partial_r = (1/r)\partial_x$ and $\partial_r^2 =
-(1/r^2)\partial_x + (1/r^2)\partial_x^2$.  This means that
$\partial_r \partial_r^\dagger = (1/r^2)\partial_x^2 - (1/r^2)$ and
$\partial_r^\dagger \partial_r = (1/r^2)\partial_x^2$.

The equations will be written with centered differences.  The quantity
$\beta_{r,j}$ indicates the value of $\beta_r$ at the $j$th
gridpoint. $\Delta x$ indicates the size of the grid step.

The linearized quantities are expected to have a time behavior that
goes as $e^{\gamma t}$. Comparing to the time behavior for the form of
a wave traveling in the positive direction,
$e^{i({\bf k} \cdot {\bf x} - \omega t)}$, we see that
$\mathrm{Re}\{\gamma\} = \mathrm{Im}\{\omega\}$, and
$\mathrm{Im}\{\gamma\} = -\mathrm{Re}\{\omega\}$. So the growth rate of the
mode is given by $\mathrm{Re}\{\gamma\}$, and the oscillation frequency is
given by $-\mathrm{Im}\{\gamma\}$.

The following finite difference equations are found:
\begin{align}
\gamma\beta_{r, j} = &-(\frac{2\eta}{r^2 \Delta x^2} + \frac{\eta}{r^2} + \frac{\eta m^2}{r^2} + \eta k^2 + im\Omega )\beta_{r, j}
\\ \nonumber
&+\frac{\eta}{r^2 \Delta x^2}(\beta_{r, j+1} + \beta_{r, j-1}) - \frac{2im \eta}{r^2}\beta_{\theta,j} + k v_A \phi_{r,j}
\end{align}
\begin{align}
\gamma\beta_{\theta,j}=&-(\frac{2\eta}{r^2 \Delta x^2} + \frac{\eta}{r^2} + \frac{\eta m^2}{r^2} + \eta k^2 + im\Omega)\beta_{\theta, j}
\\ \nonumber
&+\frac{\eta}{r^2 \Delta x^2}(\beta_{\theta, j+1} + \beta_{\theta, j-1}) +\left[\frac{\Omega_{j+1}-\Omega_{j-1}}{2\Delta x} + \frac{2im\eta}{r^2}\right]\beta_{r,j}
\\ \nonumber
&+kv_A\phi_{\theta,j}
\end{align}
\begin{align}
\gamma\phi_{r, j} = &-(\frac{2\nu}{r^2 \Delta x^2} + \frac{\nu}{r^2} + \frac{\nu m^2}{r^2} + \nu k^2 + im\Omega)\phi_{r, j}
\\ \nonumber
&+\frac{\nu}{r^2 \Delta x^2}(\phi_{r, j+1} + \phi_{r, j-1}) + \left[2\Omega_{j}-\frac{2im\nu}{r^2} \right]\phi_{\theta, j} 
\\ \nonumber
& - kv_A\beta_{r,j} - \frac{1}{2 r \Delta x}\left(\Pi_{j+1}-\Pi_{j-1}\right)
\end{align}
\begin{align}
\gamma\phi_{\theta, j} = &-(\frac{2\nu}{r^2 \Delta x^2} + \frac{\nu}{r^2} + \frac{\nu m^2}{r^2} + \nu k^2 + im\Omega)\phi_{\theta, j}
\\ \nonumber
&+\frac{\nu}{r^2 \Delta x^2}(\phi_{\theta, j+1} + \phi_{\theta, j-1}) + \left[\frac{2im\nu}{r^2} - 2\Omega_j - \frac{1}{2r\Delta x}\left(r_{j+1}\Omega_{j+1}-r_{j-1}\Omega_{j-1}\right)\right]\phi_{r,j}
\\ \nonumber
& - kv_A \beta_{\theta, j} - \frac{im}{r}\Pi_{j}
\end{align}
\begin{align}
&\left[\frac{2im\Omega_j}{r} + \frac{im}{r\Delta x}(\Omega_{j+1}-\Omega_{j-1})\right]\phi_{r,j} - \frac{2\Omega_j}{r}\phi_{\theta,j} 
\\ \nonumber
&- \frac{1}{r\Delta x}\left(\Omega_{j+1}\phi_{\theta,j+1}-\Omega_{j-1}\phi_{\theta,j-1}\right) - \left[\frac{2}{r^2\Delta x^2} + \frac{m^2}{r^2}+k^2\right]\Pi_{j}
\\ \nonumber
&+\frac{1}{r^2\Delta x^2}\left(\Pi_{j+1} + \Pi_{j-1}\right) = 0
\end{align}

\section{Boundary Conditions}

Boundary conditions are implemented by a set of constraint equations
at the inner and outer cylinders. The constraint equations are scaled
so that the terms in the equations are of the same order as the other
non-zero elements in the matrix.

\subsection{Hydrodynamic boundary conditions}

The hydrodynamic boundary conditions on $\phi_r$ and $\phi_\theta$ are
satisfied by $\phi_r = 0$ (no inflow) and $\phi_\theta = 0$ (no slip)
at the boundary.  There is a further no-slip constraint on $\phi_z$.
Because $\phi_z$ is not being solved for in the equations, this is
satisfied by setting $\partial \phi_r/\partial r = 0$ at the boundary,
implying $\phi_z=0$ by the incompressibility constraint.

\subsection{Magnetic boundary conditions}

There are two choices for the magnetic boundary condition at the inner
and outer cylinder: perfectly conducting boundaries, and perfectly
insulating boundaries.  For the perfectly conducting boundary
condition, we demand $\beta_r = 0$ and $\partial (r
\beta_\theta)/\partial r = 0$ at the wall, implying no field
perpendicular to the conductor, and no tangential component of the
current density.

The perfect insulating boundary condition is implemented as follows:
In the insulating region outside the fluid, the current density is
zero, so the magnetic field can be written in terms of a scalar
potential
\begin{equation}
{\bf B} = {\bf \nabla}\Phi
\end{equation}
Since ${\bf \nabla}\cdot{\bf B} = 0$, we know that $\nabla^2 \Phi = 0$.  In cylindrical coordinates, this means
\begin{equation}
\left[\frac{1}{r}\frac{\partial}{\partial r}r\frac{\partial}{\partial r} + \frac{1}{r^2}\frac{\partial^2}{\partial\theta^2} +\frac{\partial^2}{\partial z^2}\right]\Phi = 0
\end{equation}
If we assume a form for $\Phi$ that goes as $\Phi(r,\theta,z) = e^{i(m\theta + k z)}\tilde{\Phi}(r)$
this yields a differential equation in $\tilde{\Phi}$
\begin{equation}
\frac{d^2}{d r^2}\tilde{\Phi} + \frac{1}{r}\frac{d}{dr}\tilde{\Phi} - \left(\frac{m^2}{r^2} + k^2\right)\tilde{\Phi} = 0
\end{equation}
This equation is satisfied by modified Bessel fuctions
\begin{equation}
\tilde{\Phi}(r) = A_{in} I_m (k r) + A_{out} K_m (k r)
\end{equation}
Because of the behavior of the modified Bessel fuctions as $k r
\rightarrow 0$ and $k r \rightarrow \infty$, $A_{out} = 0$ at the
inner boundary, and $A_{in} = 0$ at the outer boundary.

The values of $B_r$ and $B_\theta$ in the ghost zone are simply found
by matching the insulating solution to the value of $B_r$ and
$B_\theta$ in the last cell of the fluid volume.

\section{Solving the equations}

These equations form a generalized eigenvalue problem which can be
solved with off-the-shelf linear algebra routines to find the
eigenvalues and eigenvectors.  The matrix that is formed by the 5N
equations (where N is the number of gridpoints) is a band diagonal
matrix with bandwidth 15, with the ordering of the variables going
$\beta_{r,0}$, $\beta_{\theta,0}$, $\phi_{r,0}$, $\phi_{\theta,0}$,
$\Pi_0$, $\beta_{r,1}$, $\beta_{\theta,1}$,...

The number of gridpoints necessary to properly resolve features on the
viscous scale is on the order of 4000, yielding a 20000x20000 matrix.
Solving for all of the eigenvalues of this system would require
prohibitive amounts of computation time.  And in any case, many of the
eigenvalues correspond to extremely strongly damped, uninteresting
modes.  \verb@ARPACK@ is used in shift-invert mode to obtain some a
subset of the eigenvalues.  It takes about 30 minutes to obtain 600
eigenvalues and eigenvectors on a desktop PC.

The choice of $\sigma$, the eigenvalue shift, is very important for
correct behavior of the code.  $\sigma$ is chosen by an iterative
technique.  $\sigma$ is intially chosen to be 10 times the inner
cylinder speed, in order to assure that it will be larger than the
fastest growing mode.  The system is solved for the first few fastest
growing modes with a large tolerance.  This only takes a few seconds.
A step is taken 4/5ths of the way from the old $\sigma$ toward the
fastest growing eigenvalue determined in the previous step.  The
tolerance is tightened, and the problem is solved again.  This is
repeated several times to find a $\sigma$ that is close to the fastest
growing eigenvalue.

It is important that $\sigma$ not be too close to the eigenvalue.
Runs in which $\sigma$ was set to be exactly the eigenvalue from the
last iteration of the $\sigma$-finding routine resulted in numerical
problems, specifically the generation of spurious eigenvalues.

\section{Benchmarking}

\subsection{Narrow gap: Non-axisymmetric Alfv\'en waves}

This test case was run in a narrow gap with the inner cylinder at 20.2
cm, and the outer cylinder at 20.3 cm.  There was no background
azimuthal flow.  The applied field was 1000 Gauss, and $\eta$ and
$\nu$ were both 3.4e-3 cm$^2$/sec.  $k$ was 17.9 1/cm, and m was 1.
Several eigenmodes, with different effective values of $k_r$ are shown
in figure~\ref{fig:narrowgapalfveneigenmodes}.  The value of $k_r$ was
calculated for each mode, and the growth rate and oscillation
frequency are plotted again $k_r$ to yield the dispersion relation for
these modes.  The dispersion relation is plotted in
figure~\ref{fig:narrowgapalfven}.

\begin{figure}
\begin{center}
\includegraphics[width=0.80\textwidth]{narrowgap_alfven_eigenmodes.eps}
\caption{Alfv\'en eigenmodes with Pm=1 in a narrow gap. Modes with
  several different values of $k_r$ are shown.}
\label{fig:narrowgapalfveneigenmodes}
\end{center}
\end{figure}

\begin{figure}
\begin{center}
\includegraphics[width=0.80\textwidth]{narrowgap_alfven_pm1.0.eps}
\caption{Alfv\'en waves with Pm=1 in a narrow gap.  The lines indicate
  the analytic solution to the dispersion relation.}
\label{fig:narrowgapalfven}
\end{center}
\end{figure}

\subsection{Narrow gap: Non-axisymmetric Alfv\'en waves with Pm=0.1}

The same problem as in the previous section was repeated, but this
time with $\nu$ of 3.4e-3 cm$^2$/sec, to yield Pm=0.1.  In this case,
the modes stop oscillating at a sufficiently high $k_r$.  The
dispersion relation is plotted in
figure~\ref{fig:narrowgapalfvenpm0.1}.

\begin{figure}
\begin{center}
\includegraphics[width=0.80\textwidth]{narrowgap_alfven_pm0.1.eps}
\caption{Alfv\'en waves with Pm=0.1 in a narrow gap.  The lines
  indicate the analytic solution to the dispersion relation.}
\label{fig:narrowgapalfvenpm0.1}
\end{center}
\end{figure}

\subsection{Narrow gap: Non-axisymmetric inertial waves}
Inertial waves were examined in the narrow gap.  In this case, the
background flow rotated at a rate of 42 radians/sec, and there was no
background magnetic field.  $k_z$ was 1795 1/cm, and $\eta$ and $\nu$
were both 3.4e-3 cm$^2$/sec.  The dispersion relation for these modes
in shown in figure~\ref{fig:narrowgapinertial}.  The result again
matches the analytic dispersion relation.  The zero frequency modes in
this case correspond to zero-frequency alfven waves.  These waves are
completely decoupled from the inertial modes since $v_{A}=0$.

\begin{figure}
\begin{center}
\includegraphics[width=0.80\textwidth]{narrowgap_inertial.eps}
\caption{Inertial waves in a narrow gap.}
\label{fig:narrowgapinertial}
\end{center}
\end{figure}

\subsection{Axisymmetric MRI modes}

The code was used to solve the problem of the slightly unstable
eigenmode from the paper of Goodman and Ji.  In this case, the inner
cylinder radius is 5cm, the outer cylinder radius is 15cm, $k$ is 0.05
1/cm, the applied field is 3000 gauss, and $\rho$, $\nu$, and $\theta$
of the fluid are those of liquid gallium: 6.0 g/cm$^3$, 3.2e-3
cm$^2$/sec, and 2.0e3 cm$^2$/sec.  The magnetic boundary condition is
insulating, and the background rotation profile is an ideal Couette
profile with $\Omega$ of 290 rad/sec at the inner cylinder, and 35
rad/sec at the outer cylinder.

The spectrum of eigenvalues for the axisymmetric case, m=0, is shown
in figure~\ref{fig:goodmanm0eigenvalues}.  The unstable mode is shown
in figure~\ref{fig:goodmanm0unstablemode}.  This mode matches that
found in Goodman and Ji's paper.

The unstable mode has a $k_r$ such that a half-wavelength of the mode
spans the gap between the two cylinders.  The other, damped modes of
the system are modes with increasing $k_r$ as they are increasingly
damped.  A couple of typical eigenmodes, with two different values of
$k_r$, are shown in figure~\ref{fig:goodmanm0typicalmodes}.

\begin{figure}
\begin{center}
\includegraphics[width=0.80\textwidth]{goodman_m0_eigenvalues.eps}
\caption{Spectrum of eigenvalues for axisymmetric problem.  Eigenvalue
  with a slightly positive growth rate and zero frequency is the MRI
  mode.}
\label{fig:goodmanm0eigenvalues}
\end{center}
\end{figure}


\begin{figure}
\begin{center}
\includegraphics[width=0.80\textwidth]{goodman_m0_unstable_mode.eps}
\caption{Unstable MRI mode}
\label{fig:goodmanm0unstablemode}
\end{center}
\end{figure}

\begin{figure}
\begin{center}
\includegraphics[width=0.80\textwidth]{goodman_m0_typical_modes.eps}
\caption{Typical modes of the axisymmetric eigenvalue problem.  The
  modes span the gap with different values of $k_r$.}
\label{fig:goodmanm0typicalmodes}
\end{center}
\end{figure}

\subsection{Nonaxisymmetric modes}

Nonaxisymmetric modes of the MRI problem were also examined.  The
spectrum of eigenvalues for m=1 is shown in
figure~\ref{fig:goodmanm1eigenvalues}.  The nonaxisymmetric modes in
this case are all damped.  Typical eigenmodes of this problem are
shown in figure~\ref{fig:goodmanm1typicalmodes}.  The modes in this
case tend to be localized at some radius.  The oscillation frequency
of the mode is Doppler shifted by the fluid retation frequency at that
radius.  The shape of the velocity component of the modes is that of a
wavepacket.  But the magnetic modes often have tails that have a
finite value across the cylinder gap.

\begin{figure}
\begin{center}
\includegraphics[width=0.80\textwidth]{goodman_m1_eigenvalues.eps}
\caption{Spectrum of eigenvalues for nonaxisymmetric problem, m=1.}
\label{fig:goodmanm1eigenvalues}
\end{center}
\end{figure}

\begin{figure}
\begin{center}
\includegraphics[width=0.80\textwidth]{goodman_m1_typical_modes.eps}
\caption{Typical modes of the nonaxisymmetric eigenvalue problem.
  Each mode is localized near a specific radius.}
\label{fig:goodmanm1typicalmodes}
\end{center}
\end{figure}


\section{References}
1. M Nornberg, H. Ji, E. Schartman, A. Roach, and J. Goodman, Phys
Rev. Lett. {\bf 104}, 074501 (2010).

\noindent 2. J. Goodman and H. Ji, J. Fluid Mech. {\bf 462}, 365 (2002).
\end{document}
